\documentclass{article}

\begin{document}

\title{Problem Set 1}
\author{Dylan Pederson}

\maketitle

\section{Problem 1}

Prove the triangle inequality
$$|A+B| \leq |A| + |B| $$

An equivalent problem is
$$ |A+B|^2 \leq (|A|+|B|)^2 $$
or $$ |A+B|^2 \leq |A|^2+|B|^2+2|A||B| $$

$$ \underline{Proof}:  $$
From the LHS:
$$ |A+B|^2 = |A|^2 + |B|^2 + <A,B> + <B,A> = |A|^2 + |B|^2 + 2<A,B> $$
And using Cauchy's inequality ($<A,B> \leq |A||B|$)
So, $$ |A+B|^2 \leq |A|^2+|B|^2+2|A||B| $$

Generalizing the triangle inequality
$$ |A_1 + A_2 + ... + A_n| \leq |A_1| + |A_2| + ... |A_n| $$
If we call
$$ B = A_1 + A_2 + ... A_{n-1} $$
then the generalized inequality becomes the normal inequality
$$|B+A_n| \leq |B| + |A_n| $$
and we can bound $|B|$ on the RHS by unrolling the sum one term at a time

\section{Problem 2}
Prove that if $\lim_{x \rightarrow a}f(x) = A$ and $\lim_{x \rightarrow a} g(x) = B$ then $\lim_{x \rightarrow a} f(x)g(x) = AB$

$$\underline{Proof}: $$
$$ [f(x)-A][g(x)-B] = f(x)g(x)-Bf(x)-Ag(x)+AB $$
$$ f(x)g(x) = [f(x)-A][g(x)-B] + Bf(x) + Ag(x) - AB $$
$$ \lim_{x \rightarrow a}f(x)g(x) = \lim_{x \rightarrow a}([f(x)-A][g(x)-B] + Bf(x) + Ag(x) - AB) $$
$$ \lim_{x \rightarrow a}f(x)g(x) = \lim_{x \rightarrow a} ([f(x)-A][g(x)-B]) + AB $$

We now pay attention to the remaining limit \\
By definition of a limit
$$ |f(x)-A| < \epsilon $$ if $ 0 < |x-a| < \delta_1$
$$ |g(x)-B| < \epsilon $$ if $ 0 < |x-a| < \delta_2$
Now if we choose $ \delta = min{\delta_1, \delta_2}$
then if $ 0 < |x-a| < \delta$
$$ |(f(x)-A)(g(x)-B)-0| \leq |f(x)-A||g(x)-B| \leq \sqrt{\epsilon}\sqrt{\epsilon} $$


\end{document}
